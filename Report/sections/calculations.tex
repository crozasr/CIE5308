\chapter{Design Calculations}

\section{Armour Layer}
For this group it is stated to build a breakwater with an armour layer of cubes. To calculate the cube size and weight of this armour layer the Van der Meer equation (1988) on the website cress\footnote{http://www.cress.nl/Regel.aspx} will be used. The input values, the equation and the damage categories can be found in the appendix under Design Armour Layer \ref{tab:design_armour}. Basically there are two material for cubes: Concrete and Rock. But due to the dimensions and the limited availability of big size rocks the cubes need to be made of concrete. The result of the calculation with Cress for the concrete cubes gives a Dn of 2.14 meter and a weight of 25 tonnes.
\section{Toe}

\section{Overtopping}

\section{Run-Up}

\section{Wave Transmission}

\section{Future Quay Wall}
The design has to take into account the presence of a future quay wall.
This puts requirements on the space available behind the breakwater and the design of the landward side of the breakwater, in such a way, that it does not hinder the future construction of a quay wall.
Especially due to the space requirement a vertical wall would be the best option in terms of the future quay wall (for instance to place a quay wall on piles, put sheetpiles and backfill them etc.).
Using a caisson or a concrete wall at the backside of the breakwater might be as costly as any other option, but more severe damage is expected due to an earthquake, which counteracts the demand for a 100 years lifetime and a minimum of maintenance, so these options are not considered further.
